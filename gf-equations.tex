\documentclass[journal=jacsat]{achemso}

%%%%%%%%%%%%%%%%%%%%%%%
% Includes various packages for extra functionality
%%%%%%%%%%%%%%%%%%%%%%%
%\usepackage{modiagram}	% Molecular orbital diagrams

\usepackage{mathtools}		% Make pretty formulas
\usepackage{fixltx2e}		% Allow subscripts
\usepackage{graphicx}		% Improved graphics
\usepackage{bm} 			% Bold math symbols 
\usepackage{txfonts} 		% Allow times-like fonts in equations
\usepackage{mathpazo} 		% Palatino fonts for equations
\usepackage{dcolumn}		% Align tabular columns on decimal points
\usepackage{footmisc}		% Foot note options
\usepackage{mathrsfs}		% RSFS fonts in equations
\usepackage{mhchem}
\usepackage{braket}
\usepackage{commath}
\usepackage[T1]{fontenc}
\usepackage{currvita}
\numberwithin{equation}{section}

%\usepackage[polish]{babel}	% Select language standard
%\usepackage{natbib}		% Support for various natbib BibTex styles


% This is for use of Times New Roman
\usepackage[T1]{fontenc}
\usepackage{times}

\newcommand{\T}[4]{T^{#1 #2}_{#3 #4}}
\renewcommand{\O}[1]{$\mathcal{O}\left( #1 \right)$}

\renewcommand{\thefootnote}{\fnsymbol{footnote}}
\renewcommand{\baselinestretch}{2} % This is for the use of double space
%\setkeys{acs}{articletitle = true} % Include title in bibliography

%%%%%%%%%%%%%%%%%%%%%%%
% Start document
%%%%%%%%%%%%%%%%%%%%%%%

%%%%%%%%%%%%%%%%%%%%%%%
% Title page
%%%%%%%%%%%%%%%%%%%%%%%

\author{Fabijan Pavo\v{s}evi\'{c}}
\email{fabijan@vt.edu}
\affiliation[Virginia Tech]
       {Department of Chemistry, Virginia Tech, Blacksburg, VA}
\title{Programmable equations of some GF methodologies}
\date{September 6, 2013}

\SectionNumbersOn
\begin{document}

\maketitle
%%%%%%%%%%%%%%%%%%%%%%%
% Paper
%%%%%%%%%%%%%%%%%%%%%%%
\newpage
\tableofcontents
\newpage
\section{General}
In order to calculate ionization energy or electron affinity $E$ for electron in orbital $p$, one need to solve the following equation iteratively:
\begin{align}
\epsilon_{p} + \Sigma_{pp}(E) = E
\end{align}
where $\epsilon_{p}$ is Hartree-Fock orbital energy of orbital $p$ while $\Sigma_{pp}(E)$ is diagonal form of the self-energy. The self-energy my include different orders in perturbative expansion and here we include programmable equations for the most common expressions such as D2 and P3 in both spin-orbital and spin-free form. Throughout this text, we use the tensor notation, thus for standard two electron integrals, we use:
\begin{align}
g^{pq}_{rs} = \braket{pq|rs}
\end{align}
where $\bar{g}^{pq}_{rs} = \bra{pq}\ket{rs}$ is anti-symmetrized form of the two electron integrals.
As usual, occupied and unoccupied orbitals are denoted by $i,j,k...$ and $a,b,c...$, respectively, while any orbital in the OBS is denoted by $p,q,r,s$.

\subsection{D2 equations in spin-orbital basis}
The expressions for the second -order self energy (D2) equations in the spin-orbital basis:
\begin{align}
\Sigma_{pq}^{(2)} = \frac{1}{2}\sum_{iab}\frac{\bar{g}^{pi}_{ab}\bar{g}^{ab}_{pi}}{E + \epsilon_{i}-\epsilon_{a}-\epsilon_{b}}+
\frac{1}{2}\sum_{ija}\frac{\bar{g}^{pa}_{ij}\bar{g}^{ij}_{pa}}{E + \epsilon_{a}-\epsilon_{i}-\epsilon_{j}}
\end{align}

\subsection{Spin-free D2 equations}
The expressions for the second -order self energy (D2) equations in the spin-free form:
\begin{align}
\Sigma_{pq}^{(2)} = \sum_{iab}\frac{g^{pi}_{ab}(2g^{ab}_{pi}-g^{ab}_{ip})}{E + \epsilon_{i}-\epsilon_{a}-\epsilon_{b}}+
\sum_{ija}\frac{g^{pa}_{ij}(2g^{ij}_{pa}-g^{ij}_{ap})}{E + \epsilon_{a}-\epsilon_{i}-\epsilon_{j}}
\end{align}

\subsection{P3 equations in spin-orbital basis}
The expressions for the partial third-order self energy (P3) equations in the spin-orbital basis:
\begin{align}
\Sigma_{pq}^{(\text{P}3)} =\frac{1}{2}\sum_{iab}\frac{\bar{g}^{pi}_{ab}\bar{g}^{ab}_{qi}}{E + \epsilon_{i}-\epsilon_{a}-\epsilon_{b}}+\frac{1}{2}\sum_{ija}\frac{\bar{g}^{pa}_{ij}W_{qaij}}{E + \epsilon_{a}-\epsilon_{i}-\epsilon_{j}}+\frac{1}{2}\sum_{ija}\frac{U_{paij}(E)\bar{g}^{ij}_{qa}}{E + \epsilon_{a}-\epsilon_{i}-\epsilon_{j}}
\end{align}
with,
\begin{align}
W_{qaij}=\bar{g}^{qa}_{ij}+\frac{1}{2}\sum_{bc}\frac{\bar{g}^{qa}_{bc}\bar{g}^{bc}_{ij}}{\epsilon_{i} + \epsilon_{j}-\epsilon_{b}-\epsilon_{c}}+(1-P_{ij})\sum_{bk}\frac{\bar{g}^{qk}_{bi}\bar{g}^{ba}_{jk}}{\epsilon_{j} + \epsilon_{k}-\epsilon_{a}-\epsilon_{b}}
\end{align}
\begin{align}
U_{qaij}(E)=-\frac{1}{2}\sum_{kl}\frac{\bar{g}^{pa}_{kl}\bar{g}^{kl}_{ij}}{E + \epsilon_{a}-\epsilon_{k}-\epsilon_{l}}-(1-P_{ij})\sum_{bk}\frac{\bar{g}^{pb}_{jk}\bar{g}^{ak}_{bi}}{E + \epsilon_{b}-\epsilon_{j}-\epsilon_{k}}
\end{align}
where $P_{ij}$ is symmetrizer.

\subsection{Spin-free P3 equations}
The expressions for the partial third-order self energy (P3) equations in the spin-free form:
\begin{align}
\Sigma_{pq}^{(\text{P}3)} =\sum_{iab}\frac{g^{pi}_{ab}(2g^{ab}_{pi}-g^{ab}_{ip})}{E + \epsilon_{i}-\epsilon_{a}-\epsilon_{b}}+\sum_{ija}\frac{g^{pa}_{ij}W_{qaij}}{E + \epsilon_{a}-\epsilon_{i}-\epsilon_{j}}+\sum_{ija}\frac{U_{paij}(E)g^{ij}_{qa}}{E + \epsilon_{a}-\epsilon_{i}-\epsilon_{j}}
\end{align}
with 
\begin{align}
W_{qaij}&=2g^{qa}_{ij}-g^{qa}_{ji}+\sum_{bc}\frac{g^{qa}_{bc}(2g^{bc}_{ij}-g^{bc}_{ji})}{\epsilon_{i} + \epsilon_{j}-\epsilon_{b}-\epsilon_{c}} \\\nonumber
&+\frac{1}{2}\sum_{bk}\frac{g^{qk}_{bi}g^{ba}_{jk}-2g^{qk}_{bi}g^{ba}_{kj}-2g^{qk}_{ib}g^{ba}_{jk}+4g^{qk}_{ib}g^{ba}_{kj}}{\epsilon_{j} + \epsilon_{k}-\epsilon_{a}-\epsilon_{b}} \\\nonumber
&+\frac{1}{2}\sum_{bk}\frac{-2g^{qk}_{bj}g^{ba}_{ik}+g^{qk}_{bj}g^{ba}_{ki}+g^{qk}_{jb}g^{ba}_{ik}-2g^{qk}_{jb}g^{ba}_{ki}}{\epsilon_{i} + \epsilon_{k}-\epsilon_{a}-\epsilon_{b}} \\\nonumber
&+\frac{1}{2}\sum_{bk}\frac{-2g^{qk}_{bj}g^{ba}_{ik}+g^{qk}_{bj}g^{ba}_{ki}+g^{qk}_{jb}g^{ba}_{ik}-2g^{qk}_{jb}g^{ba}_{ki}}{\epsilon_{i} + \epsilon_{k}-\epsilon_{a}-\epsilon_{b}} \\\nonumber
&+\frac{1}{2}\sum_{bk}\frac{g^{qk}_{bi}g^{ba}_{jk}-2g^{qk}_{bi}g^{ba}_{kj}-2g^{qk}_{ib}g^{ba}_{jk}+4g^{qk}_{ib}g^{ba}_{kj}}{\epsilon_{j} + \epsilon_{k}-\epsilon_{a}-\epsilon_{b}}
\end{align}

\begin{align}
U_{paij}(E)&=-\sum_{kl}\frac{g^{pa}_{kl}(2g^{kl}_{ij}-g^{kl}_{ji})}{E + \epsilon_{a}-\epsilon_{k}-\epsilon_{l}} \\\nonumber
&-\frac{1}{2}\sum_{bk}\frac{g^{pb}_{jk}g^{ak}_{bi}-2g^{pb}_{jk}g^{ak}_{ib}-2g^{pb}_{kj}g^{ak}_{bi}+g^{pb}_{kj}g^{ak}_{ib}}{E + \epsilon_{b}-\epsilon_{j}-\epsilon_{k}} \\\nonumber
&-\frac{1}{2}\sum_{bk}\frac{-2g^{pb}_{ik}g^{ak}_{bj}+4g^{pb}_{ik}g^{ak}_{jb}+g^{pb}_{ki}g^{ak}_{bj}-2g^{pb}_{ki}g^{ak}_{jb}}{E + \epsilon_{b}-\epsilon_{i}-\epsilon_{k}} \\\nonumber
&-\frac{1}{2}\sum_{bk}\frac{-2g^{pb}_{ik}g^{ak}_{bj}+4g^{pb}_{ik}g^{ak}_{jb}+g^{pb}_{ki}g^{ak}_{bj}-2g^{pb}_{ki}g^{ak}_{jb}}{E + \epsilon_{b}-\epsilon_{i}-\epsilon_{k}} \\\nonumber
&-\frac{1}{2}\sum_{bk}\frac{g^{pb}_{jk}g^{ak}_{bi}-2g^{pb}_{jk}g^{ak}_{ib}-2g^{pb}_{kj}g^{ak}_{bi}+g^{pb}_{kj}g^{ak}_{ib}}{E + \epsilon_{b}-\epsilon_{j}-\epsilon_{k}}
\end{align}
Both expressions for $W$ and $U$ matrix elements can be reduced furthermore.

\end{document}
